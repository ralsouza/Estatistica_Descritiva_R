\documentclass[portuguese,]{article}
\usepackage{lmodern}
\usepackage{amssymb,amsmath}
\usepackage{ifxetex,ifluatex}
\usepackage{fixltx2e} % provides \textsubscript
\ifnum 0\ifxetex 1\fi\ifluatex 1\fi=0 % if pdftex
  \usepackage[T1]{fontenc}
  \usepackage[utf8]{inputenc}
\else % if luatex or xelatex
  \ifxetex
    \usepackage{mathspec}
  \else
    \usepackage{fontspec}
  \fi
  \defaultfontfeatures{Ligatures=TeX,Scale=MatchLowercase}
\fi
% use upquote if available, for straight quotes in verbatim environments
\IfFileExists{upquote.sty}{\usepackage{upquote}}{}
% use microtype if available
\IfFileExists{microtype.sty}{%
\usepackage{microtype}
\UseMicrotypeSet[protrusion]{basicmath} % disable protrusion for tt fonts
}{}
\usepackage[margin=1in]{geometry}
\usepackage{hyperref}
\hypersetup{unicode=true,
            pdftitle={Estatística Descritiva no Software R},
            pdfauthor={Nome: Rafael Lima de Souza; RU: 1237272; Polo: Porto Alegre - Centro Histórico; e-mail: ls.rafael@icloud.com},
            pdfborder={0 0 0},
            breaklinks=true}
\urlstyle{same}  % don't use monospace font for urls
\ifnum 0\ifxetex 1\fi\ifluatex 1\fi=0 % if pdftex
  \usepackage[shorthands=off,main=portuguese]{babel}
\else
  \usepackage{polyglossia}
  \setmainlanguage[]{portuguese}
\fi
\usepackage{color}
\usepackage{fancyvrb}
\newcommand{\VerbBar}{|}
\newcommand{\VERB}{\Verb[commandchars=\\\{\}]}
\DefineVerbatimEnvironment{Highlighting}{Verbatim}{commandchars=\\\{\}}
% Add ',fontsize=\small' for more characters per line
\usepackage{framed}
\definecolor{shadecolor}{RGB}{248,248,248}
\newenvironment{Shaded}{\begin{snugshade}}{\end{snugshade}}
\newcommand{\KeywordTok}[1]{\textcolor[rgb]{0.13,0.29,0.53}{\textbf{#1}}}
\newcommand{\DataTypeTok}[1]{\textcolor[rgb]{0.13,0.29,0.53}{#1}}
\newcommand{\DecValTok}[1]{\textcolor[rgb]{0.00,0.00,0.81}{#1}}
\newcommand{\BaseNTok}[1]{\textcolor[rgb]{0.00,0.00,0.81}{#1}}
\newcommand{\FloatTok}[1]{\textcolor[rgb]{0.00,0.00,0.81}{#1}}
\newcommand{\ConstantTok}[1]{\textcolor[rgb]{0.00,0.00,0.00}{#1}}
\newcommand{\CharTok}[1]{\textcolor[rgb]{0.31,0.60,0.02}{#1}}
\newcommand{\SpecialCharTok}[1]{\textcolor[rgb]{0.00,0.00,0.00}{#1}}
\newcommand{\StringTok}[1]{\textcolor[rgb]{0.31,0.60,0.02}{#1}}
\newcommand{\VerbatimStringTok}[1]{\textcolor[rgb]{0.31,0.60,0.02}{#1}}
\newcommand{\SpecialStringTok}[1]{\textcolor[rgb]{0.31,0.60,0.02}{#1}}
\newcommand{\ImportTok}[1]{#1}
\newcommand{\CommentTok}[1]{\textcolor[rgb]{0.56,0.35,0.01}{\textit{#1}}}
\newcommand{\DocumentationTok}[1]{\textcolor[rgb]{0.56,0.35,0.01}{\textbf{\textit{#1}}}}
\newcommand{\AnnotationTok}[1]{\textcolor[rgb]{0.56,0.35,0.01}{\textbf{\textit{#1}}}}
\newcommand{\CommentVarTok}[1]{\textcolor[rgb]{0.56,0.35,0.01}{\textbf{\textit{#1}}}}
\newcommand{\OtherTok}[1]{\textcolor[rgb]{0.56,0.35,0.01}{#1}}
\newcommand{\FunctionTok}[1]{\textcolor[rgb]{0.00,0.00,0.00}{#1}}
\newcommand{\VariableTok}[1]{\textcolor[rgb]{0.00,0.00,0.00}{#1}}
\newcommand{\ControlFlowTok}[1]{\textcolor[rgb]{0.13,0.29,0.53}{\textbf{#1}}}
\newcommand{\OperatorTok}[1]{\textcolor[rgb]{0.81,0.36,0.00}{\textbf{#1}}}
\newcommand{\BuiltInTok}[1]{#1}
\newcommand{\ExtensionTok}[1]{#1}
\newcommand{\PreprocessorTok}[1]{\textcolor[rgb]{0.56,0.35,0.01}{\textit{#1}}}
\newcommand{\AttributeTok}[1]{\textcolor[rgb]{0.77,0.63,0.00}{#1}}
\newcommand{\RegionMarkerTok}[1]{#1}
\newcommand{\InformationTok}[1]{\textcolor[rgb]{0.56,0.35,0.01}{\textbf{\textit{#1}}}}
\newcommand{\WarningTok}[1]{\textcolor[rgb]{0.56,0.35,0.01}{\textbf{\textit{#1}}}}
\newcommand{\AlertTok}[1]{\textcolor[rgb]{0.94,0.16,0.16}{#1}}
\newcommand{\ErrorTok}[1]{\textcolor[rgb]{0.64,0.00,0.00}{\textbf{#1}}}
\newcommand{\NormalTok}[1]{#1}
\usepackage{graphicx,grffile}
\makeatletter
\def\maxwidth{\ifdim\Gin@nat@width>\linewidth\linewidth\else\Gin@nat@width\fi}
\def\maxheight{\ifdim\Gin@nat@height>\textheight\textheight\else\Gin@nat@height\fi}
\makeatother
% Scale images if necessary, so that they will not overflow the page
% margins by default, and it is still possible to overwrite the defaults
% using explicit options in \includegraphics[width, height, ...]{}
\setkeys{Gin}{width=\maxwidth,height=\maxheight,keepaspectratio}
\IfFileExists{parskip.sty}{%
\usepackage{parskip}
}{% else
\setlength{\parindent}{0pt}
\setlength{\parskip}{6pt plus 2pt minus 1pt}
}
\setlength{\emergencystretch}{3em}  % prevent overfull lines
\providecommand{\tightlist}{%
  \setlength{\itemsep}{0pt}\setlength{\parskip}{0pt}}
\setcounter{secnumdepth}{0}
% Redefines (sub)paragraphs to behave more like sections
\ifx\paragraph\undefined\else
\let\oldparagraph\paragraph
\renewcommand{\paragraph}[1]{\oldparagraph{#1}\mbox{}}
\fi
\ifx\subparagraph\undefined\else
\let\oldsubparagraph\subparagraph
\renewcommand{\subparagraph}[1]{\oldsubparagraph{#1}\mbox{}}
\fi

%%% Use protect on footnotes to avoid problems with footnotes in titles
\let\rmarkdownfootnote\footnote%
\def\footnote{\protect\rmarkdownfootnote}

%%% Change title format to be more compact
\usepackage{titling}

% Create subtitle command for use in maketitle
\newcommand{\subtitle}[1]{
  \posttitle{
    \begin{center}\large#1\end{center}
    }
}

\setlength{\droptitle}{-2em}

  \title{Estatística Descritiva no Software R}
    \pretitle{\vspace{\droptitle}\centering\huge}
  \posttitle{\par}
    \author{Nome: Rafael Lima de Souza \\ RU: 1237272 \\ Polo: Porto Alegre - Centro Histórico \\ e-mail:
\href{mailto:ls.rafael@icloud.com}{\nolinkurl{ls.rafael@icloud.com}}}
    \preauthor{\centering\large\emph}
  \postauthor{\par}
      \predate{\centering\large\emph}
  \postdate{\par}
    \date{25 janeiro 2019}

\usepackage{booktabs}
\usepackage{longtable}
\usepackage{array}
\usepackage{multirow}
\usepackage{wrapfig}
\usepackage{float}
\usepackage{colortbl}
\usepackage{pdflscape}
\usepackage{tabu}
\usepackage{threeparttable}
\usepackage{threeparttablex}
\usepackage[normalem]{ulem}
\usepackage{makecell}
\usepackage{xcolor}

\begin{document}
\maketitle

\section{Introdução Sobre o R}\label{introducao-sobre-o-r}

O R é uma linguagem de programação estatística, trata-se de uma
linguagem de programação especializada em computação de dados. Algumas
de suas principais características são o seu caráter gratuito e sua
disponibilidade para uma gama bastante variada de sistemas operacionais.
É também altamente expansível com o uso de pacotes que são bibliotecas
para áreas de estudos ou funções específicas, onde é possível executar
cálculos complexos e ainda gerar uma infinidade de gráficos.

Ainda é possível contar com um
\href{https://pt.wikipedia.org/wiki/Ambiente_de_desenvolvimento_integrado}{ambiente
de desenvolvimento integrado}, o
\href{https://www.rstudio.com}{RStudio}, onde é possível gerar
relatórios e apresentações com alto nível de qualidade. Inclusive, este
trabalho foi totalmente desenvolvido com o editor
\href{https://rmarkdown.rstudio.com}{R Markdown}.

O código fonte desde trabalho podem ser encontrado no repositório
\href{https://github.com/ralsouza/Estatistica_Descritiva_R}{Estatística
Descritiva com R} no \href{https://github.com}{GitHub} e o trabalho
destaca os links que podem ser acessados para o leitor ver os detalhes
dos termos usados neste trabalho.

\subsubsection{Medição dos Dados}\label{medicao-dos-dados}

Para os cálculos deste relatório, usamos as funções nativas do R, que
foram desenvolvidas préviamente pelos desenvolvedores. Abaixo
analisaremos as medidas de tendência central conhecidas por
\href{https://pt.wikipedia.org/wiki/Mediana_(estat\%C3\%ADstica)}{Mediana},
\href{https://pt.wikipedia.org/wiki/M\%C3\%A9dia}{Média} e as medidas de
dispersão chamadas de
\href{https://pt.wikipedia.org/wiki/Vari\%C3\%A2ncia}{Variância} e o
\href{https://pt.wikipedia.org/wiki/Desvio_padr\%C3\%A3o}{Desvio
Padrão}.

\paragraph{Mediana}\label{mediana}

A mediana de um conjunto de dados é o valor que ocupa a posição central,
desde que estejam colocados em ordem crescente ou decrescente, ou seja,
em um rol.

A fórmula da Mediana é: \(Md = \frac{n+1}{2}\), desta forma é possível
encontrar a posição da mediana.

No R, a função que calcula a Mediana é chamada de
\href{https://www.rdocumentation.org/packages/stats/versions/3.5.2/topics/median}{median},
abaixo mostra o nome da coluna calculada após o símbolo \#, depois o
comando usado para o cálculo e na sequencia ao lado do símbolo \#\#
mostra o resultado calculado.

\paragraph{Média}\label{media}

A média aritimética simples, ou simplesmente média, nada mais é que a
soma dos resultados obtidos dividida pela quantidade dos resultados.

A fórmula da Média é: \(\ \overline{X} = \frac{\sum{X_i}}{n}\), em que
\(i\) varia de 1 até \(n\).

A função utilizada para calcular a Média no R, é chamada de
\href{https://www.rdocumentation.org/packages/base/versions/3.5.2/topics/mean}{mean}.

\paragraph{Variância}\label{variancia}

Como a soma dos desvios em relação à média é sempre igual a zero, é
possível evitar este fato elevando cada desvio ao quadrado, pois sabemos
que o quadrado de qualquer número real é sempre positivo.

Para calcular a variância de uma população, usamos a seguinte fórmula:
\(S^2=\frac{\sum[(X_i-\ \overline{X})^2.f_i]}{N}\)

Quando se trata de calcular uma amostra, a seguinte fórmula deve ser
aplicada: \(S^2=\frac{\sum[(X_i-\ \overline{X})^2.f_i]}{N-1}\)

A função utilizada para calcular a Variância no R, é chamada de
\href{https://www.rdocumentation.org/packages/cmvnorm/versions/1.0-5/topics/var}{var}.

\paragraph{Desvio Padrão}\label{desvio-padrao}

O desvio padrão o quanto os dados estão dispersos em torno da média. Um
desvio padrão alto, indica que os dados estão espalhados por uma ampla
gama de valores. O desvio padrão, nada mais é do que a raíz quadrada da
variância.

Sua fórmula para medir a variabilidade dos dados em uma população, é:
\(S^2=\sqrt{\frac{\sum[(X_i-\ \overline{X})^2.f_i]}{N}}\)

E para fazermos a medição em uma amostra, usamos:
\(S^2=\sqrt{\frac{\sum[(X_i-\ \overline{X})^2.f_i]}{N-1}}\)

A função utilizada para calcular o Desvio Padrão no R, é chamada de
\href{https://www.rdocumentation.org/packages/stats/versions/3.5.2/topics/sd}{sd}.

\subsection{Experimento 1 - Acidentes de
Trânsito}\label{experimento-1---acidentes-de-transito}

A tabela abaixo mostra o número de acidentes de trânsito durante uma
semana em uma grande metrópole.

\begin{Shaded}
\begin{Highlighting}[]
\CommentTok{# Criação de um objeto do tipo matriz(7x4), para armazenar os dados dos acidentes.}
\NormalTok{mt_acidentes <-}\StringTok{ }\KeywordTok{matrix}\NormalTok{(}\KeywordTok{c}\NormalTok{(}\DecValTok{70}\NormalTok{,}\DecValTok{42}\NormalTok{,}\DecValTok{45}\NormalTok{,}\DecValTok{42}\NormalTok{,}\DecValTok{50}\NormalTok{,}\DecValTok{61}\NormalTok{,}\DecValTok{72}\NormalTok{,}\DecValTok{25}\NormalTok{,}\DecValTok{15}\NormalTok{,}\DecValTok{22}\NormalTok{,}\DecValTok{23}\NormalTok{,}\DecValTok{24}\NormalTok{,}\DecValTok{36}\NormalTok{,}\DecValTok{40}\NormalTok{,}
                         \DecValTok{75}\NormalTok{,}\DecValTok{54}\NormalTok{,}\DecValTok{32}\NormalTok{,}\DecValTok{30}\NormalTok{,}\DecValTok{42}\NormalTok{,}\DecValTok{50}\NormalTok{,}\DecValTok{52}\NormalTok{,}\DecValTok{20}\NormalTok{,}\DecValTok{5}\NormalTok{ ,}\DecValTok{6}\NormalTok{ ,}\DecValTok{5}\NormalTok{ ,}\DecValTok{8}\NormalTok{ ,}\DecValTok{15}\NormalTok{,}\DecValTok{18}\NormalTok{),}
                       \DataTypeTok{nrow =} \DecValTok{7}\NormalTok{,}
                       \DataTypeTok{ncol =} \DecValTok{4}\NormalTok{)}

\CommentTok{# Nomeando os nomes das variáveis e observações. }
\KeywordTok{row.names}\NormalTok{(mt_acidentes) <-}\StringTok{ }\KeywordTok{c}\NormalTok{(}\StringTok{'Domingo'}\NormalTok{,}\StringTok{'Segunda'}\NormalTok{,}\StringTok{'Terça'}\NormalTok{,}\StringTok{'Quarta'}\NormalTok{,}\StringTok{'Quinta'}\NormalTok{,}\StringTok{'Sexta'}\NormalTok{,}\StringTok{'Sábado'}\NormalTok{)}
\KeywordTok{colnames}\NormalTok{(mt_acidentes) <-}\StringTok{ }\KeywordTok{c}\NormalTok{(}\StringTok{'Sem Vítimas'}\NormalTok{,}\StringTok{'Com Ferimentos Graves'}\NormalTok{,}\StringTok{'Com Ferimentos Leves'}\NormalTok{,}\StringTok{'Com Mortos'}\NormalTok{)}
\end{Highlighting}
\end{Shaded}

\subsubsection{Resultado do Experimento
1}\label{resultado-do-experimento-1}

Como resultado dos comandos acima, obtemos a seguinte tabela com os
dados sobre os acidentes ocorridos durante uma semana.

\begin{table}[t]

\caption{\label{tab:tabela acidentes}Tabela 1 - Número de Acidentes de Trânsito na Metrópole A}
\centering
\begin{tabular}{l|c|c|c|c}
\hline
  & Sem Vítimas & Com Ferimentos Graves & Com Ferimentos Leves & Com Mortos\\
\hline
Domingo & 70 & 25 & 75 & 20\\
\hline
Segunda & 42 & 15 & 54 & 5\\
\hline
Terça & 45 & 22 & 32 & 6\\
\hline
Quarta & 42 & 23 & 30 & 5\\
\hline
Quinta & 50 & 24 & 42 & 8\\
\hline
Sexta & 61 & 36 & 50 & 15\\
\hline
Sábado & 72 & 40 & 52 & 18\\
\hline
\multicolumn{5}{l}{\textit{Fonte:}}\\
\multicolumn{5}{l}{Autor do portifolio.}\\
\end{tabular}
\end{table}

\subsubsection{Análise dos Dados de
Acidentes}\label{analise-dos-dados-de-acidentes}

Abaixo segue o código usado para calcular as medidas dos dados dos
acidentes.

\begin{Shaded}
\begin{Highlighting}[]
\CommentTok{# Comando usado para calcular as medidas e armazenar os dados em um dataframe}
\NormalTok{df_ferimentos_graves <-}\StringTok{  }\KeywordTok{data.frame}\NormalTok{(}\DataTypeTok{tipo    =} \KeywordTok{c}\NormalTok{(}\StringTok{'Sem Vítimas'}\NormalTok{,}\StringTok{'Com Ferimentos Graves'}\NormalTok{,}\StringTok{'Com Ferimentos Leves'}\NormalTok{,}\StringTok{'Com Mortos'}\NormalTok{),}
                                     \DataTypeTok{media   =} \KeywordTok{apply}\NormalTok{(mt_acidentes,}\DecValTok{2}\NormalTok{, mean),}
                                     \DataTypeTok{des.padr  =} \KeywordTok{apply}\NormalTok{(mt_acidentes,}\DecValTok{2}\NormalTok{, sd),}
                                     \DataTypeTok{var     =} \KeywordTok{apply}\NormalTok{(mt_acidentes,}\DecValTok{2}\NormalTok{, var),}
                                     \DataTypeTok{mediana =} \KeywordTok{apply}\NormalTok{(mt_acidentes,}\DecValTok{2}\NormalTok{, median))}
\CommentTok{# Remove os nomes das linhas}
\KeywordTok{row.names}\NormalTok{(df_ferimentos_graves) <-}\StringTok{ }\OtherTok{NULL}
\end{Highlighting}
\end{Shaded}

Podemos constatar abaixo, que a média dos feridos gravemente foi de 26 e
o desvio padrão foi 8. Este desvio padrão indica que existe boa
aderência da variabilidade dos dados com a média.

\begin{table}[t]

\caption{\label{tab:tabela acidentes calculada}Resultados das Medidas Centrais e Dispersão}
\centering
\begin{tabular}{c|c|c|c|c}
\hline
tipo & media & des.padr & var & mediana\\
\hline
Sem Vítimas & 54.57143 & 12.985340 & 168.61905 & 50\\
\hline
Com Ferimentos Graves & 26.42857 & 8.618916 & 74.28571 & 24\\
\hline
Com Ferimentos Leves & 47.85714 & 15.279928 & 233.47619 & 50\\
\hline
Com Mortos & 11.00000 & 6.480741 & 42.00000 & 8\\
\hline
\end{tabular}
\end{table}

No seguinte gráfico podemos analisar visualmente os dados de cada tipo
de acidente e constatar que o desvio padrão possui relativa aderência à
média.

\begin{Shaded}
\begin{Highlighting}[]
\CommentTok{# Biblioteca GGPLOT2 para visualização de dados}
\KeywordTok{ggplot}\NormalTok{(df_ferimentos_graves, }\KeywordTok{aes}\NormalTok{(}\DataTypeTok{x=}\NormalTok{tipo, }\DataTypeTok{y=}\NormalTok{media)) }\OperatorTok{+}\StringTok{ }
\StringTok{  }\KeywordTok{geom_errorbar}\NormalTok{(}\KeywordTok{aes}\NormalTok{(}\DataTypeTok{ymin=}\NormalTok{media}\OperatorTok{-}\NormalTok{des.padr, }\DataTypeTok{ymax=}\NormalTok{media}\OperatorTok{+}\NormalTok{des.padr), }\DataTypeTok{width=}\NormalTok{.}\DecValTok{2}\NormalTok{) }\OperatorTok{+}
\StringTok{  }\KeywordTok{geom_line}\NormalTok{() }\OperatorTok{+}
\StringTok{  }\KeywordTok{geom_point}\NormalTok{()}
\end{Highlighting}
\end{Shaded}

\begin{center}\includegraphics{Estatistica_Descritiva_R_files/figure-latex/teste-1} \end{center}

\subsection{Experimento 2 - Gráficos Bioma
Pampa}\label{experimento-2---graficos-bioma-pampa}

\begin{Shaded}
\begin{Highlighting}[]
\CommentTok{# Criação de um objeto do tipo matriz(7x4), para armazenar os dados dos acidentes.}
\NormalTok{mt_bioma <-}\StringTok{ }\KeywordTok{matrix}\NormalTok{(}\KeywordTok{c}\NormalTok{(}\DecValTok{176496}\NormalTok{, }\DecValTok{6210}\NormalTok{, }\DecValTok{3340}\NormalTok{, }\DecValTok{1607}\NormalTok{, }\DecValTok{122682}\NormalTok{, }\DecValTok{20974}\NormalTok{, }\DecValTok{7658}\NormalTok{, }\DecValTok{14025}\NormalTok{,}
                         \DecValTok{103835}\NormalTok{, }\DecValTok{0}\NormalTok{   , }\DecValTok{0}\NormalTok{   , }\DecValTok{428}\NormalTok{ , }\DecValTok{10980}\NormalTok{ , }\DecValTok{2033}\NormalTok{ , }\DecValTok{394}\NormalTok{ , }\DecValTok{0}\NormalTok{    ,}
                         \DecValTok{58636}\NormalTok{ , }\DecValTok{6210}\NormalTok{, }\DecValTok{3340}\NormalTok{, }\DecValTok{1179}\NormalTok{, }\DecValTok{21702}\NormalTok{ , }\DecValTok{18940}\NormalTok{, }\DecValTok{7264}\NormalTok{, }\DecValTok{0}\NormalTok{   ),}
                       \DataTypeTok{nrow =} \DecValTok{8}\NormalTok{,}
                       \DataTypeTok{ncol =} \DecValTok{3}\NormalTok{)}

\CommentTok{# Nomeando os nomes das variáveis e observações. }
\KeywordTok{row.names}\NormalTok{(mt_bioma) <-}\StringTok{ }\KeywordTok{c}\NormalTok{(}\StringTok{'Área Total'}\NormalTok{,}
                             \StringTok{'Floresta Estacional Semidecidual'}\NormalTok{,}
                             \StringTok{'Floresta Estacional Decidual'}\NormalTok{,}
                             \StringTok{'Savana Estépica'}\NormalTok{,}
                             \StringTok{'Estepe'}\NormalTok{,}
                             \StringTok{'Formações pioneiras'}\NormalTok{,}
                             \StringTok{'Contatos entre tipos de vegetação'}\NormalTok{,}
                             \StringTok{'Superfície com água'}\NormalTok{)}
\KeywordTok{colnames}\NormalTok{(mt_bioma) <-}\StringTok{ }\KeywordTok{c}\NormalTok{(}\StringTok{'Total do bioma (Km²)'}\NormalTok{,}\StringTok{'Área remanescente (Km²)'}\NormalTok{,}\StringTok{'Área antroponizada (Km²)'}\NormalTok{)}
\end{Highlighting}
\end{Shaded}

\begin{table}[t]

\caption{\label{tab:tabela bioma}Tabela 2 - Áreas Remanescentes e Áreas Antropizadas, no Bioma Pampa, Segundo os Tipos de Vegetação}
\centering
\begin{tabular}{l|c|c|c}
\hline
  & Total do bioma (Km²) & Área remanescente (Km²) & Área antroponizada (Km²)\\
\hline
Área Total & 176496 & 103835 & 58636\\
\hline
Floresta Estacional Semidecidual & 6210 & 0 & 6210\\
\hline
Floresta Estacional Decidual & 3340 & 0 & 3340\\
\hline
Savana Estépica & 1607 & 428 & 1179\\
\hline
Estepe & 122682 & 10980 & 21702\\
\hline
Formações pioneiras & 20974 & 2033 & 18940\\
\hline
Contatos entre tipos de vegetação & 7658 & 394 & 7264\\
\hline
Superfície com água & 14025 & 0 & 0\\
\hline
\multicolumn{4}{l}{\textit{Fonte:}}\\
\multicolumn{4}{l}{FIGUEIREDO, 2016}\\
\end{tabular}
\end{table}

\paragraph{Apresentação dos Gráficos das Áreas
Antropizadas}\label{apresentacao-dos-graficos-das-areas-antropizadas}


\end{document}
